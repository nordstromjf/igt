In this chapter we will look at a specific type of two-player game. These are often the first games studied in game theory as they can be straightforward to analyze. All of our games in this chapter will have only two players. We will also focus on games in which one player's win is the other player's loss. 

\section{Introduction to Two-Person Zero-Sum Games}\index{zero-sum game}

%\markright{Two Player Zero-Sum Games}

\vspace{.2in}


In all of the examples from the last section, whatever one player won, the other player lost. 

\begin{definition}\label{D:zerosum} A two player game is called a \emph{zero-sum}\index{zero-sum game} game if the sum of the payoffs to each player is constant for all possible outcomes of the game. More specifically, the terms (or coordinates) in each payoff vector must add up to the same value for each payoff vector. Such games are sometimes called\emph{constant-sum}\index{constant-sum game} games instead.
\end{definition}

We can always think of zero-sum games as being games in which one player's win is the other player's loss.

\begin{example}\label{poker}
Consider a poker game in which each player comes to the game with \$100. If there are five players, then the sum of money for all five players is always \$500. At any give time during the game, a particular player may have more than \$100, but then another player must have less than \$100. One player's win is another player's loss.
\end{example}

\begin{example}\label{E:cakecutting2} Consider the cake division game. Determine the payoff matrix for this game. It is important to determine what each player's options are first: how can the ``cutter'' cut the cake? How can the ``chooser''  pick her piece? The payoff matrix is given in Table \ref{T:cakecutting}.

%\hspace{1.2in}Chooser

%\begin{tabular}{l|r|c|c|}\cline{2-4}
%&&\textbf{Large Piece}&\textbf{Small Piece}\\ \cline{2-4}
%Cutter&\textbf{Cut Evenly} &(half, half)&(half, half)\\ \cline{2-4}
%&\textbf{Cut Unevenly} &(small, large)&(large, small)\\ \cline{2-4}

%\end{tabular}

\begin{table}[h]
\centering

\begin{tabular}{cccc}
                      & \multicolumn{3}{c}{Chooser}                                                  \\ \cline{2-4} 
\multicolumn{1}{l|}{} & \multicolumn{1}{l|}{} & \multicolumn{1}{c|}{Larger Piece} & \multicolumn{1}{c|}{Smaller Piece} \\ \cline{2-4} 
\multicolumn{1}{l|}{Cutter} & \multicolumn{1}{c|}{Cut Evenly} & \multicolumn{1}{c|}{(half, half)} & \multicolumn{1}{c|}{(half, half)} \\ \cline{2-4} 
\multicolumn{1}{l|}{} & \multicolumn{1}{c|}{Cut Unvenly} & \multicolumn{1}{c|}{(small piece, large piece)} & \multicolumn{1}{c|}{(large piece, small piece)} \\ \cline{2-4} 
\end{tabular}
\caption{Payoff matrix for Cake Cutting game}
\label{T:cakecutting}
\end{table}

%\medskip

In order to better see that this game is zero-sum (or constant-sum), we could give values for the amount of cake each player gets. For example, half the cake would be 50\%, a small piece might be 40\%. Then we can rewrite the matrix with the percentage values in Table \ref{T:cakecuttingpercent}

%\hspace{1.2in}Chooser

%\begin{tabular}{l|r|c|c|}\cline{2-4}
%&&\textbf{Large Piece}&\textbf{Small Piece}\\ \cline{2-4}
%Cutter&\textbf{Cut Evenly} &(50, 50)&(50, 50)\\ \cline{2-4}
%&\textbf{Cut Unevenly} &(40, 60)&(60, 40)\\ \cline{2-4}

%\end{tabular}
%\medskip

\begin{table}[h]
\centering

\begin{tabular}{cccc}
                      & \multicolumn{3}{c}{Chooser}                                                  \\ \cline{2-4} 
\multicolumn{1}{l|}{} & \multicolumn{1}{l|}{} & \multicolumn{1}{c|}{Larger Piece} & \multicolumn{1}{c|}{Smaller Piece} \\ \cline{2-4} 
\multicolumn{1}{l|}{Cutter} & \multicolumn{1}{c|}{Cut Evenly} & \multicolumn{1}{c|}{$(50, 50)$} & \multicolumn{1}{c|}{$(50, 50)$} \\ \cline{2-4} 
\multicolumn{1}{l|}{} & \multicolumn{1}{c|}{Cut Unvenly} & \multicolumn{1}{c|}{$(40, 60)$} & \multicolumn{1}{c|}{$(60, 40)$} \\ \cline{2-4} 
\end{tabular}
\caption{Payoff matrix, in percent of cake, for Cake Cutting game}
\label{T:cakecuttingpercent}
\end{table}



In each outcome, the payoffs to each player add up to 100 (or 100\%). In more mathematical terms, the coordinates of each payoff vector add up to 100. Thus the sum is the same, or constant, for each outcome. 
\end{example}

It is probably simple to see from the matrix in Table \ref{T:cakecuttingpercent} that Player 2 will always choose the large piece, thus Player 1 does best to cut the cake evenly. The outcome of the game is the \emph{strategy pair}\index{strategy pair} denoted \{Cut Evenly, Larger Piece\}, with resulting payoff vector $(50, 50)$.


But why are we going to call these games ``zero-sum'' rather than ``constant-sum''?  We can convert any zero-sum game to a game where the payoffs actually sum to zero.

\begin{example}\label{E:pokerzero} Consider the above poker game where each player begins the game with \$100. Suppose at some point in the game  the five players have the following amounts of money: \$50, \$200, \$140, \$100. \$10. Then we could think of their gain as -\$50, \$100, \$40, \$0, -\$90. What do these five numbers add up to?
\end{example}

\begin{example}\label{E:cakecuttingzero} Convert the cake division payoffs so that the payoff vectors sum to zero (rather than 100). 

The solution is given in Table \ref{T:cakecuttingzerosum}.

\begin{table}[h]
\centering

\begin{tabular}{cccc}
                      & \multicolumn{3}{c}{Chooser}                                                  \\ \cline{2-4} 
\multicolumn{1}{l|}{} & \multicolumn{1}{l|}{} & \multicolumn{1}{c|}{Larger Piece} & \multicolumn{1}{c|}{Smaller Piece} \\ \cline{2-4} 
\multicolumn{1}{l|}{Cutter} & \multicolumn{1}{c|}{Cut Evenly} & \multicolumn{1}{c|}{$(0, 0)$} & \multicolumn{1}{c|}{$(0, 0)$} \\ \cline{2-4} 
\multicolumn{1}{l|}{} & \multicolumn{1}{c|}{Cut Unvenly} & \multicolumn{1}{c|}{$(-10, 10)$} & \multicolumn{1}{c|}{$(10, -10)$} \\ \cline{2-4} 
\end{tabular}
\caption{Zero-sum payoff matrix for Cake Cutting game}
\label{T:cakecuttingzerosum}
\end{table}


%\hspace{1.2in}Chooser

%\begin{tabular}{l|r|c|c|}\cline{2-4}
%&&\textbf{Large Piece}&\textbf{Small Piece}\\ \cline{2-4}
%Cutter&\textbf{Cut Evenly} &(0, 0)&(0, 0)\\ \cline{2-4}
%&\textbf{Cut Unevenly} &(-10, 10)&(10, -10)\\ \cline{2-4}

%\end{tabular}

%\medskip
But let's make sure we understand what these numbers mean. For example, a payoff of $(0,0)$ does not mean each player gets no cake, it means they don't get any more cake than the other player. In this example, each player gets half the cake (50\%) plus the payoff.
\end{example}

\bigskip

In the form of Example \ref{E:cakecuttingzero}, it is easy to recognize a zero-sum game since each payoff vector has the form $(a, -a)$ (or $(-a, a)$).

 