
%\section{Prisoner's Dilemma and Chicken}

%\markright{Prisoner's Dilemma}

%\vspace{.2in}
%Do all of your work on your own paper. Give complete answers (complete sentences!).


In order to better understand non-zero-sum games we look at two classic games. 

%The first game is called \emph{Prisoner's Dilemma}\index{Prisoner's Dilemma}. 

\begin{example}\label{E:PrisonersDilemma}\textbf{Prisoner's Dilemma.}\index{Prisoner's Dilemma}
Two partners in crime are arrested for  burglary and sent to separate rooms. They are each offered a deal: if they confess and rat on their partner, they will receive a reduced sentence. So if one confesses and the other doesn't, the confessor only gets 3 months in prison, while the partner serves 10 years. If both confess, then they each get 8 years. However, if neither confess, there isn't enough evidence, and each gets just one year. We can represent the situation with the matrix in Table \ref{T:PrisonersDilemma}.

%\hspace{3in}Prisoner 2

%\begin{center}
%\begin{tabular}{l|r|c|c|}\cline{2-4}
%&&\textbf{Confess}&\textbf{Don't confess}\\ \cline{2-4}
%Prisoner 1&\textbf{Confess} &(8, 8)&(0.25, 10)\\ \cline{2-4}
%&\textbf{Don't confess} &(10, 0.25)&(1, 1)\\ \cline{2-4}
%\end{tabular}
%\end{center}
%\vspace{.1in}

\begin{table}[h]
\centering

\begin{tabular}{cccc}
                      & \multicolumn{3}{c}{Prisoner 2}                                                  \\ \cline{2-4} 
\multicolumn{1}{l|}{} & \multicolumn{1}{l|}{} & \multicolumn{1}{c|}{Confess} & \multicolumn{1}{c|}{Don't Confess} \\ \cline{2-4} 
\multicolumn{1}{l|}{Prisoner 1} & \multicolumn{1}{c|}{Confess} & \multicolumn{1}{c|}{$(8, 8)$} & \multicolumn{1}{c|}{$(0.25, 10)$} \\ \cline{2-4} 
\multicolumn{1}{l|}{} & \multicolumn{1}{c|}{Don't Confess} & \multicolumn{1}{c|}{$(10, 0.25)$} & \multicolumn{1}{c|}{$(1, 1)$} \\ \cline{2-4} 
\end{tabular}
\caption{Prisoner's Dilemma (years in prison)}
\label{T:PrisonersDilemma}
\end{table}
\end{example}


\begin{xca}\label{E:PDdomstrat}
Does the matrix in Table \ref{T:PrisonersDilemma} have any dominated strategies for Player 1? Does it have any dominated strategies for Player 2? Keep in mind that a prisoner prefers smaller numbers since prison time is bad.
\end{xca}

\begin{xca}\label{E:PDbeststrat}
Suppose you are Prisoner 1. What should you do? Why? Suppose you are Prisoner 2. What should you do? Why? Does your choice of strategies result in an equilibrium pair?
\end{xca}

\begin{xca}\label{E:PDbestforall}
Looking at the game as an outsider, what strategy pair is the best option for both prisoners. 
\end{xca}

\begin{xca}\label{E:PDsamedecision}
Now suppose both prisoners are perfectly rational, so that any decision Prisoner 1 makes would also be the decision Prisoner 2 makes. Further, suppose both prisoners know that their opponent is perfectly rational. What should each prisoner do?
\end{xca}

\begin{xca}\label{E:PDrandomP2}
Suppose Prisoner 2 is crazy and is likely to confess with 50/50 chance. What should Prisoner 1 do? Does it change if he confesses with a 75\% chance? What if he confesses with a 25\% chance.
\end{xca}

\begin{xca}\label{E:PDcommunicate}
Suppose the prisoners are able to communicate about their strategy. How might this affect what they choose to do?
\end{xca}

\begin{xca}\label{E:PDdilemma}
Explain why this is a ``dilemma'' for the prisoners. Is it likely they will chose a strategy which leads to the best outcome for both? You might want to consider whether there are dominated strategies.
\end{xca}

\begin{example}\label{E:Chicken}\textbf{Chicken.}\index{Chicken}
Two drivers drive towards each other. If one driver swerves, he is considered a ``chicken.'' If a driver doesn't swerve (drives straight), he is considered the winner. Of course if neither swerves, they crash and neither wins. A possible payoff matrix for this game is given in Table \ref{T:chicken}

%\hspace{3in}Driver 2

%\begin{center}
%\begin{tabular}{l|r|c|c|}\cline{2-4}
%&&\textbf{Swerve}&\textbf{Straight}\\ \cline{2-4}
%Driver 1&\textbf{Swerve} &(0, 0)&(-1, 10)\\ \cline{2-4}
%&\textbf{Straight} &(10, -1)&(-100, -100)\\ \cline{2-4}
%\end{tabular}
%\end{center}
%\vspace{.1in}

\begin{table}[h]
\centering

\begin{tabular}{cccc}
                      & \multicolumn{3}{c}{Driver 2}                                                  \\ \cline{2-4} 
\multicolumn{1}{l|}{} & \multicolumn{1}{l|}{} & \multicolumn{1}{c|}{Swerve} & \multicolumn{1}{c|}{Straight} \\ \cline{2-4} 
\multicolumn{1}{l|}{Driver 1} & \multicolumn{1}{c|}{Swerve} & \multicolumn{1}{c|}{$(0, 0)$} & \multicolumn{1}{c|}{$(-1, 10)$} \\ \cline{2-4} 
\multicolumn{1}{l|}{} & \multicolumn{1}{c|}{Straight} & \multicolumn{1}{c|}{$(10, -1)$} & \multicolumn{1}{c|}{$(-100, -100)$} \\ \cline{2-4} 
\end{tabular}
\caption{Chicken}
\label{T:chicken}
\end{table}
\end{example}


%\begin{enumerate}
%\setcounter{enumi}{7}

\begin{xca}\label{E:Cdomstrat}
Does game in Table \ref{T:chicken} have any dominated strategies?
\end{xca}

\begin{xca}\label{E:Cbestoutcome}
What strategy results in the best outcome for Driver 1? What strategy results in the best outcome for Driver 2? What happens if they both choose that strategy?
\end{xca}

\begin{xca}\label{E:Cequilpairs}
Consider the strategy pair with outcome $(-1, 10)$. Does Driver 1 have any regrets about his choice? What about Driver 2? Is this an equilibrium pair? Are there any other points in which neither driver would regret his choice?
\end{xca}

\begin{xca}\label{E:Cbeststrat}
Can you determine what each driver should do in this game? If so, does this result in an equilibrium pair?
\end{xca}

\begin{xca}\label{E:Csamestrat}
Now suppose both drivers in the game of chicken are perfectly rational, so that any decision Driver 1 makes would also be the decision Driver 2 makes. Further, suppose both drivers know that their opponent is perfectly rational. What should each driver do?
\end{xca}

\begin{xca}\label{E:Crandom} Suppose Driver 2 is a remote control dummy and will swerve or drive straight with a 50/50 chance. What should Driver 1 do? Does it change if he swerves with 75\% chance?
\end{xca}

\begin{xca}\label{E:Ccommunicate}
Can it benefit drivers in the game of chicken to communicate about their strategy? Explain.
\end{xca}

\begin{xca}\label{E:comparePDC} 
Compare Prisoner's Dilemma and Chicken. Are there dominated strategies in both games? Are there equilibrium pairs? Are players likely to reach the optimal payoff for one player, both players, or neither player? Does a player's choice depend on what he knows about his opponent (perfectly rational or perfectly random)?
\end{xca}





 