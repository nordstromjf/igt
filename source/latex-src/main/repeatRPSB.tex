%\section*{ Introduction to Repeated Two-person Zero-Sum Games}

%\markright{Rock-Paper-Scissors}
%\textbf{Rock-Paper-Scissors.}\index{Rock-Paper-Scissors}

%\vspace{3.5in}
%Do all of your work on your own paper. Give complete answers (complete sentences!).

%In this section, we will examine strategies for two-person zero-sum games when played repeatedly. 


Recall the familiar game \textbf{Rock-Paper-Scissors}: ROCK beats SCISSORS, SCISSORS beat PAPER, and PAPER beats ROCK.

\begin{xca}\label{E:RPSmatrix}
Construct a game matrix for Rock-Paper-Scissors (RPS).
\end{xca}

\begin{xca}\label{E:RPSzs} 
Is RPS a zero-sum game? Does it have an equilibrium point? Explain.
\end{xca}

\begin{xca}\label{E:RPSplay}
We want to look at what happens if we repeat this game. Play the game ten times with an opponent. Record the results (list strategy pairs and payoffs for each player).
\end{xca}

\begin{xca}\label{E:RPSconjstrat}
Describe any strategy you used.
\end{xca}

\begin{xca}\label{E:RPSreflect}
Reflect on your chosen strategy. Does it guarantee you a ``win''? What should it mean to ``win'' in a repeated game? What are the strengths and weaknesses of you strategy? 
\end{xca}

\begin{xca}\label{E:RPSshare}
Discuss your strategy with someone else in the class (it can be your opponent). After sharing your ideas for strategy, can you improve your previous strategy?
\end{xca}



Although you may have come up with a good strategy, let's see if we can't decide what the ``best'' strategy should be for RPS. Let's assume we are playing RPS against the smartest player to ever live. We will call such an opponent the ``perfect'' player.



%\begin{enumerate}
%\setcounter{enumi}{6}

\begin{xca}\label{E:RPSpure}
Explain why it is not a good idea to play a pure strategy; i.e. to play only ROCK, only PAPER, or only SCISSORS.
\end{xca}

\begin{xca}\label{E:RPSuneven}
Does it make sense to play one option more often than another (for example, ROCK more often than PAPER)? Explain.
\end{xca}

\begin{xca}\label{E:RPSstratbest}
How often should you play each option?
\end{xca}

\begin{xca}\label{E:RPSpattern}
Do you want to play in a predictable pattern or randomly?  What are some advantages and disadvantages of a pattern? What are some advantages and disadvantages of a random strategy?
\end{xca}



Hopefully, you concluded that the best strategy against our perfect player would be to play ROCK, PAPER, SCISSORS 1/3 of the time each, and to play randomly. We can say that our strategy is to play each option randomly with a probability of 1/3, and call this the Random(1/3, 1/3, 1/3) strategy.

%\begin{enumerate}
%\setcounter{enumi}{10}

\begin{xca}\label{E:RPSltpayoff}
Using this ``best'' strategy, what do you predict the long term payoff will be for Player 1? For Player 2?
\end{xca}

\begin{xca}\label{E:RPSdice}
Let's check our prediction. Using a die, let 1 and 2 represent ROCK, 3 and 4 represent PAPER, and 5 and 6 represent SCISSORS. Play the game 20 times with someone in class where each player rolls to determine the choice of R, P, or S. Keep track of the strategy pairs and payoffs. What was the total payoff for each player? [At this point, if you still feel that you have a better strategy, try your strategy against the random one-- see what happens!]
\end{xca}

\begin{xca}\label{E:RPSdicecompare}
How did the actual outcome compare to your predicted outcome? What do you expect would happen if you play the game 100 times? (Or more?) 
\end{xca}


Using ideas about probability and expected value we can more precisely answer Exercise \ref{E:RPSdicecompare}.

%\begin{enumerate}
%\setcounter{enumi}{13}

\begin{xca}\label{E:RPSproboutcomes}
Assume both players are using the Random(1/3, 1/3, 1/3) strategy. List all the possible outcomes for a single game (recall the outcome is the strategy pair and the payoff, for example \{R, P\}, $(-1, 1)$). What is the probability that any particular pair of strategies will be played? Are the strategy pairs equally likely? 
\end{xca}


\begin{xca}\label{E:RPSexpval}
Using the probabilities and payoffs from Exercise \ref{E:RPSproboutcomes} calculate the expected value of the game for each player.
\end{xca}


\begin{xca}\label{E:smallrepeatstrat}
Now consider the matrix from Example \ref{E:smallrepeat} above: 
\[\left[\begin{matrix}
1&0\\
-1&2

\end{matrix}\right].
\]
See if you can determine how often Player 1 should play each row, and how often Player 2 should play each column. Try testing your proposed strategy (you may be able to use a variation on the dice as we saw above). Write up any conjectured strategies and the results from playing the game with your strategy. Do you think you have come up with the best strategy? Explain.
\end{xca}







 