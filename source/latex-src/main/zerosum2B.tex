 
%\subsection{More Two Player Zero-Sum Games}

%\markright{More Two Player Zero-Sum Games}

%\vspace{.2in}
%Do all of your work on your own paper. Give complete answers (complete sentences!).

%\vspace{.1in}


\begin{xca}\label{E:domstratpractice1}
Use the idea of eliminating dominated strategies to determine any equilibrium pairs in the zero-sum game given in Table \ref{T:domstratpractice1}. Note, since it is a zero-sum game we need only show Player 1's payoffs. Explain all the steps in your solution. If you are unable to find an equilibrium pair, explain what goes wrong.
%\vspace{.1in}

%\hspace{1in}Player 2

%\begin{tabular}{l|r|c|c|c|c|}\cline{2-6}
%&&\textbf{W}&\textbf{X}&\textbf{Y}&\textbf{Z}\\ \cline{2-6}
%Player 1&\textbf{A} &1&0&0&10\\ \cline{2-6}
%&\textbf{B} &-1&0&-2&9\\ \cline{2-6}
%&\textbf{C} &1&1&1&8\\ \cline{2-6}
%&\textbf{D} &-2&0&0&7\\ \cline{2-6}
%\end{tabular}
%\vspace{.2in}

\begin{table}[h]
\centering
\begin{tabular}{lccccc}
                      & \multicolumn{5}{c}{Player 2}                                                                                                  \\ \cline{2-6} 
\multicolumn{1}{l|}{} & \multicolumn{1}{c|}{} & \multicolumn{1}{c|}{W} & \multicolumn{1}{c|}{X} & \multicolumn{1}{c|}{Y} & \multicolumn{1}{c|}{Z} \\ \cline{2-6} 
\multicolumn{1}{l|}{Player 1} & \multicolumn{1}{c|}{A} & \multicolumn{1}{c|}{1} & \multicolumn{1}{c|}{0} & \multicolumn{1}{c|}{0} & \multicolumn{1}{c|}{0} \\ \cline{2-6} 
\multicolumn{1}{l|}{} & \multicolumn{1}{c|}{B} & \multicolumn{1}{c|}{-1} & \multicolumn{1}{c|}{0} & \multicolumn{1}{c|}{-2} & \multicolumn{1}{c|}{9} \\ \cline{2-6} 
\multicolumn{1}{l|}{} & \multicolumn{1}{c|}{C} & \multicolumn{1}{c|}{1} & \multicolumn{1}{c|}{1} & \multicolumn{1}{c|}{1} & \multicolumn{1}{c|}{8} \\ \cline{2-6} 
\multicolumn{1}{l|}{} & \multicolumn{1}{c|}{D} & \multicolumn{1}{c|}{-2} & \multicolumn{1}{c|}{0} & \multicolumn{1}{c|}{0} & \multicolumn{1}{c|}{7} \\ \cline{2-6} 
\end{tabular}
\caption{Payoff matrix for Exercise \ref{E:domstratpractice1}}
\label{T:domstratpractice1}

\end{table}
\end{xca}

\begin{xca}\label{E:domstratpractice2} Determine any equilibrium pairs in the zero-sum game given in Table \ref{T:domstratpractice2}.  Explain all the steps in your solution. If you are unable to find an equilibrium pair, explain what goes wrong.

%\vspace{.1in}

%\hspace{1in}Player 2

%\begin{tabular}{l|r|c|c|c|c|}\cline{2-6}
%&&\textbf{W}&\textbf{X}&\textbf{Y}&\textbf{Z}\\ \cline{2-6}
%Player 1&\textbf{A} &1&2&3&4\\ \cline{2-6}
%&\textbf{B} &0&-1&0&5\\ \cline{2-6}
%&\textbf{C} &-1&3&2&4\\ \cline{2-6}
%&\textbf{D} &0&1&-1&1\\ \cline{2-6}
%\end{tabular}
%\vspace{.2in}

\begin{table}[h]
\centering
\begin{tabular}{lccccc}
                      & \multicolumn{5}{c}{Player 2}                                                                                                  \\ \cline{2-6} 
\multicolumn{1}{l|}{} & \multicolumn{1}{c|}{} & \multicolumn{1}{c|}{W} & \multicolumn{1}{c|}{X} & \multicolumn{1}{c|}{Y} & \multicolumn{1}{c|}{Z} \\ \cline{2-6} 
\multicolumn{1}{l|}{Player 1} & \multicolumn{1}{c|}{A} & \multicolumn{1}{c|}{1} & \multicolumn{1}{c|}{2} & \multicolumn{1}{c|}{3} & \multicolumn{1}{c|}{4} \\ \cline{2-6} 
\multicolumn{1}{l|}{} & \multicolumn{1}{c|}{B} & \multicolumn{1}{c|}{0} & \multicolumn{1}{c|}{-1} & \multicolumn{1}{c|}{0} & \multicolumn{1}{c|}{5} \\ \cline{2-6} 
\multicolumn{1}{l|}{} & \multicolumn{1}{c|}{C} & \multicolumn{1}{c|}{-1} & \multicolumn{1}{c|}{3} & \multicolumn{1}{c|}{2} & \multicolumn{1}{c|}{4} \\ \cline{2-6} 
\multicolumn{1}{l|}{} & \multicolumn{1}{c|}{D} & \multicolumn{1}{c|}{0} & \multicolumn{1}{c|}{1} & \multicolumn{1}{c|}{-1} & \multicolumn{1}{c|}{1} \\ \cline{2-6} 
\end{tabular}
\caption{Payoff matrix for Exercise \ref{E:domstratpractice2}}
\label{T:domstratpractice2}

\end{table}
\end{xca}


\begin{xca}\label{E:domstratpractice3} Determine any equilibrium pairs in the zero-sum game given in Table \ref{T:domstratpractice3}.  Explain all the steps in your solution. If you are unable to find an equilibrium pair, explain what goes wrong.


%\vspace{.1in}

%\hspace{1in}Player 2

%\begin{tabular}{l|r|c|c|c|c|}\cline{2-6}
%&&\textbf{W}&\textbf{X}&\textbf{Y}&\textbf{Z}\\ \cline{2-6}
%Player 1&\textbf{A} &-2&0&3&20\\ \cline{2-6}
%&\textbf{B} &1&-2&-3&0\\ \cline{2-6}
%&\textbf{C} &10&-10&-1&1\\ \cline{2-6}
%&\textbf{D} &0&0&10&15\\ \cline{2-6}
%\end{tabular}
%\vspace{.2in}
\begin{table}[h]
\centering
\begin{tabular}{lccccc}
                      & \multicolumn{5}{c}{Player 2}                                                                                                  \\ \cline{2-6} 
\multicolumn{1}{l|}{} & \multicolumn{1}{c|}{} & \multicolumn{1}{c|}{W} & \multicolumn{1}{c|}{X} & \multicolumn{1}{c|}{Y} & \multicolumn{1}{c|}{Z} \\ \cline{2-6} 
\multicolumn{1}{l|}{Player 1} & \multicolumn{1}{c|}{A} & \multicolumn{1}{c|}{-2} & \multicolumn{1}{c|}{0} & \multicolumn{1}{c|}{3} & \multicolumn{1}{c|}{20} \\ \cline{2-6} 
\multicolumn{1}{l|}{} & \multicolumn{1}{c|}{B} & \multicolumn{1}{c|}{1} & \multicolumn{1}{c|}{-2} & \multicolumn{1}{c|}{-3} & \multicolumn{1}{c|}{0} \\ \cline{2-6} 
\multicolumn{1}{l|}{} & \multicolumn{1}{c|}{C} & \multicolumn{1}{c|}{10} & \multicolumn{1}{c|}{-10} & \multicolumn{1}{c|}{-1} & \multicolumn{1}{c|}{1} \\ \cline{2-6} 
\multicolumn{1}{l|}{} & \multicolumn{1}{c|}{D} & \multicolumn{1}{c|}{0} & \multicolumn{1}{c|}{0} & \multicolumn{1}{c|}{10} & \multicolumn{1}{c|}{15} \\ \cline{2-6} 
\end{tabular}
\caption{Payoff matrix for Exercise \ref{E:domstratpractice3}}
\label{T:domstratpractice3}

\end{table}
\end{xca}


\begin{xca}\label{E:domstratpractice4} Determine any equilibrium pairs in the zero-sum game given in Table \ref{T:domstratpractice4}.  Explain all the steps in your solution. If you are unable to find an equilibrium pair, explain what goes wrong.


%\vspace{.1in}

%\hspace{1in}Player 2

%\begin{tabular}{l|r|c|c|c|c|}\cline{2-6}
%&&\textbf{W}&\textbf{X}&\textbf{Y}&\textbf{Z}\\ \cline{2-6}
%Player 1&\textbf{A} &-2&0&3&20\\ \cline{2-6}
%&\textbf{B} &1&-2&-5&-3\\ \cline{2-6}
%&\textbf{C} &10&-10&-1&1\\ \cline{2-6}
%&\textbf{D} &0&0&10&8\\ \cline{2-6}
%\end{tabular}
%\vspace{.2in}

\begin{table}[h]
\centering
\begin{tabular}{lccccc}
                      & \multicolumn{5}{c}{Player 2}                                                                                                  \\ \cline{2-6} 
\multicolumn{1}{l|}{} & \multicolumn{1}{c|}{} & \multicolumn{1}{c|}{W} & \multicolumn{1}{c|}{X} & \multicolumn{1}{c|}{Y} & \multicolumn{1}{c|}{Z} \\ \cline{2-6} 
\multicolumn{1}{l|}{Player 1} & \multicolumn{1}{c|}{A} & \multicolumn{1}{c|}{-2} & \multicolumn{1}{c|}{0} & \multicolumn{1}{c|}{3} & \multicolumn{1}{c|}{20} \\ \cline{2-6} 
\multicolumn{1}{l|}{} & \multicolumn{1}{c|}{B} & \multicolumn{1}{c|}{1} & \multicolumn{1}{c|}{-2} & \multicolumn{1}{c|}{-5} & \multicolumn{1}{c|}{-3} \\ \cline{2-6} 
\multicolumn{1}{l|}{} & \multicolumn{1}{c|}{C} & \multicolumn{1}{c|}{10} & \multicolumn{1}{c|}{-10} & \multicolumn{1}{c|}{-1} & \multicolumn{1}{c|}{1} \\ \cline{2-6} 
\multicolumn{1}{l|}{} & \multicolumn{1}{c|}{D} & \multicolumn{1}{c|}{0} & \multicolumn{1}{c|}{0} & \multicolumn{1}{c|}{10} & \multicolumn{1}{c|}{8} \\ \cline{2-6} 
\end{tabular}
\caption{Payoff matrix for Exercise \ref{E:domstratpractice4}}
\label{T:domstratpractice4}

\end{table}
\end{xca}


Chances are you had trouble determining an equilibrium pair for the last game. Does this mean there isn't an equilibrium pair? Not necessarily, but we are stuck if we try to use only the idea of eliminating dominated strategies. So we need a new method. 

We might think of this as the ``worst case scenario,'' or ``extremely defensive play.'' The idea is that we want to assume our opponent is the best player to ever live. In fact, we might assume our opponent is telepathic. So no matter what we do, our opponent will always guess what we are going to choose. 

\begin{example}\label{E:intromaxmin}
Assume you are Player 1, and you are playing against this ``infinitely smart'' Player 2. Consider the game in Exercise \ref{E:domstratpractice1}. If you pick row A, what will Player 2 do? Player 2 would pick column X or Y. Try this for each of the rows. Which row is your best choice? If you pick A, you will get 0; if you pick B, you will get $-2$; if you pick C, you will get 1; and if you pick D you will get $-2$. Thus, your best choice is to choose C and get 1. Now assume you are Player 2, and Player 1 is ``infinitely smart.'' Which column is your best choice? If you pick W, Player 1 will get 1 (you will get $-1$); if you pick X, Player 1 will get $1$; if you pick Y, Player 1 will get 1; and if you pick Z you will get $10$. Thus, you can choose W, X, or Y (since you want Player 1 to win the least) and get $-1$.
\end{example}

%\begin{enumerate}
%\setcounter{enumi}{4}
\begin{xca}
Using the method described in Example \ref{E:intromaxmin}, determine what each player should do in the game in Exercise \ref{E:domstratpractice2}.
\end{xca}



\begin{xca}
Using the method described in Example \ref{E:intromaxmin}, determine what each player should do in the game in Exercise \ref{E:domstratpractice3}.
\end{xca}

\begin{xca}
Generalize this method. In other words, give a general rule for how Player 1 should determine his or her best move. Do the same for Player 2.
\end{xca}



\begin{xca}
What do you notice about using this method on the games in Exercises   \ref{E:domstratpractice1},  \ref{E:domstratpractice2}, and \ref{E:domstratpractice3}? Is the solution an equilibrium pair?
\end{xca}

\begin{xca}
Now try this method on the elusive payoff matrix in Exercise  \ref{E:domstratpractice4}. What should each player do? Do you think we get an equilibrium pair? Explain.
\end{xca}


%\vspace{.1in}

 The strategies we found using the above method have a more official name. Player~1's strategy is called the \emph{maximin}\index{maximin strategy} strategy. Player~1 is maximizing the minimum values from each row. Player 2's strategy is called the \emph{minimax}\index{minimax} strategy. Player 2 is minimizing the maximum values from each column. 




%\begin{enumerate}
%\setcounter{enumi}{9}
\begin{xca}\label{E:domstratpractice5}
Let's consider another game matrix, given in Table \ref{T:domstratpractice5}. Explain why you cannot use dominated strategies to find an equilibrium pair. Do you think there is an equilibrium pair for this game (why or why not)?



%\hspace{1in}Player 2

%\begin{tabular}{l|r|c|c|c|c|}\cline{2-6}
%&&\textbf{W}&\textbf{X}&\textbf{Y}&\textbf{Z}\\ \cline{2-6}
%Player 1&\textbf{A} &-2&0&3&20\\ \cline{2-6}
%&\textbf{B} &1&2&-3&0\\ \cline{2-6}
%&\textbf{C} &10&-10&-1&1\\ \cline{2-6}
%&\textbf{D} &0&0&10&15\\ \cline{2-6}
%\end{tabular}
%\vspace{.2in}
\begin{table}[h]
\centering
\begin{tabular}{lccccc}
                      & \multicolumn{5}{c}{Player 2}                                                                                                  \\ \cline{2-6} 
\multicolumn{1}{l|}{} & \multicolumn{1}{c|}{} & \multicolumn{1}{c|}{W} & \multicolumn{1}{c|}{X} & \multicolumn{1}{c|}{Y} & \multicolumn{1}{c|}{Z} \\ \cline{2-6} 
\multicolumn{1}{l|}{Player 1} & \multicolumn{1}{c|}{A} & \multicolumn{1}{c|}{-2} & \multicolumn{1}{c|}{0} & \multicolumn{1}{c|}{3} & \multicolumn{1}{c|}{20} \\ \cline{2-6} 
\multicolumn{1}{l|}{} & \multicolumn{1}{c|}{B} & \multicolumn{1}{c|}{1} & \multicolumn{1}{c|}{2} & \multicolumn{1}{c|}{-3} & \multicolumn{1}{c|}{0} \\ \cline{2-6} 
\multicolumn{1}{l|}{} & \multicolumn{1}{c|}{C} & \multicolumn{1}{c|}{10} & \multicolumn{1}{c|}{-10} & \multicolumn{1}{c|}{-1} & \multicolumn{1}{c|}{1} \\ \cline{2-6} 
\multicolumn{1}{l|}{} & \multicolumn{1}{c|}{D} & \multicolumn{1}{c|}{0} & \multicolumn{1}{c|}{0} & \multicolumn{1}{c|}{10} & \multicolumn{1}{c|}{15} \\ \cline{2-6} 
\end{tabular}
\caption{Payoff matrix for Exercise \ref{E:domstratpractice5}}
\label{T:domstratpractice5}

\end{table}
\end{xca}


\begin{xca} 
If both players use the maximin/ minimax strategy, what is the outcome of the game in Table \ref{T:domstratpractice5}? 
\end{xca}


\begin{xca} 
In the game in Table \ref{T:domstratpractice5}, if Player 1's opponent can guess that Player 1 will choose to use a maximin strategy, is Player 1 better off \emph{not} using the maximin strategy?
\end{xca}

\begin{xca}\label{E:iterate5}
Suppose both players initially decide to use the minimax/ maximin strategy in the game in Table \ref{T:domstratpractice5}. Is Player 1 better off choosing a different strategy? If Player 2 guesses a change, is Player 2 better off changing strategies? Continue this line of reasoning for several iterations. What strategies do each of the players choose? Is at least one player always better off switching strategies? Can we conclude that the maximin/ minimax strategy does not lead to an equilibrium pair?
\end{xca}


\begin{xca}
Compare your answers in Exercise \ref{E:iterate5} to what happens in Exercises \ref{E:domstratpractice1},  \ref{E:domstratpractice2}, and \ref{E:domstratpractice3}. Can you identify any key differences between the games in Exercise \ref{E:domstratpractice5} and Exercises \ref{E:domstratpractice1},  \ref{E:domstratpractice2}, and \ref{E:domstratpractice3}?
   
\end{xca}

 