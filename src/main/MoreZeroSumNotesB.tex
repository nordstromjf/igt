
\section{More Two-Person Zero-Sum Games: Dominated Strategies}

%\markright{More Two Player Zero-Sum Games}

\vspace{.2in}


Recall that in a zero-sum game, we know that one player's win is the other player's loss. Furthermore, we know we can rewrite any zero-sum game so that the player's payoffs are in the form $(a, -a)$. Note, this works even if $a$ is negative; in which case, $-a$ is positive.

\begin{example}\label{E:domstrat1} Consider the zero-sum game with payoff matrix in Table \ref{T:smallexample}. Note that for simplicity our payoff matrix contains only the payoffs and not the strategy names; but Player 1 still chooses a row and Player 2 still chooses a column.

%\hspace{1.2in}Player 2

%\begin{tabular}{l|c|c|}\cline{2-3}
%Player 1&(1, -1)&(0, 0)\\ \cline{2-3}
%&(-1, 1)&(-2, 2)\\ \cline{2-3}

%\end{tabular}
\begin{table}[h]
\centering

\begin{tabular}{ccc}
                      & \multicolumn{2}{c}{Player 2}                                                  \\ \cline{2-3} 
%\multicolumn{1}{l|}{} & \multicolumn{1}{l|}{} & \multicolumn{1}{c|}{H} & \multicolumn{1}{c|}{T} \\ \cline{2-4} 
\multicolumn{1}{l|}{Player 1}  & \multicolumn{1}{c|}{$(1, -1)$} & \multicolumn{1}{c|}{$(-0, 0)$} \\ \cline{2-3} 
\multicolumn{1}{l|}{}  & \multicolumn{1}{c|}{$(-1, 1)$} & \multicolumn{1}{c|}{$(-2, 2)$} \\ \cline{2-3} 
\end{tabular}
\caption{The payoff matrix for Example \ref{E:domstrat1}}
\label{T:smallexample}
\end{table}


If we know we are playing a zero-sum game, then the use of ordered pair seems somewhat redundant: If Player 1 wins 1, then we know that Player 2 must lose 1 (win $-1$). Thus, if we KNOW we are playing a zero-sum game, we can simplify our notation by just using Player 1's payoffs. The above matrix in Table \ref{T:smallexample} can be simplified as in Table \ref{T:smallexampleP1}.

%\hspace{.8in}Player 2

%\begin{tabular}{l|c|c|}\cline{2-3}
%Player 1&1&0\\ \cline{2-3}
%&-1&-2\\ \cline{2-3}

%\end{tabular}

\begin{table}[h]
\centering

\begin{tabular}{ccc}
                      & \multicolumn{2}{c}{Player 2}                                                  \\ \cline{2-3} 
%\multicolumn{1}{l|}{} & \multicolumn{1}{l|}{} & \multicolumn{1}{c|}{H} & \multicolumn{1}{c|}{T} \\ \cline{2-4} 
\multicolumn{1}{l|}{Player 1}  & \multicolumn{1}{c|}{1} & \multicolumn{1}{c|}{0} \\ \cline{2-3} 
\multicolumn{1}{l|}{}  & \multicolumn{1}{c|}{-1} & \multicolumn{1}{c|}{-2} \\ \cline{2-3} 
\end{tabular}
\caption{The payoff matrix for Example \ref{E:domstrat1} using only Player 1's payoffs}
\label{T:smallexampleP1}
\end{table}

\end{example}


When simplifying, keep a few things in mind:
\begin{enumerate}
\item You MUST know that the game is zero-sum.
\item If it is not otherwise specified, the payoffs represent Player 1's payoffs.
\item You can always give a similar matrix representing Player 2's payoffs. However, due to (2), you should indicate that the matrix is for Player 2. For example, Player 2's payoff matrix would be given by Table \ref{T:smallexampleP2}.

%\hspace{.8in}Player 2

%\begin{tabular}{l|c|c|}\cline{2-3}
%Player 1&-1&0\\ \cline{2-3}
%&1&2\\ \cline{2-3}

%\end{tabular}\ .

\begin{table}[h]
\centering

\begin{tabular}{ccc}
                      & \multicolumn{2}{c}{Player 2}                                                  \\ \cline{2-3} 
%\multicolumn{1}{l|}{} & \multicolumn{1}{l|}{} & \multicolumn{1}{c|}{H} & \multicolumn{1}{c|}{T} \\ \cline{2-4} 
\multicolumn{1}{l|}{Player 1}  & \multicolumn{1}{c|}{-1} & \multicolumn{1}{c|}{0} \\ \cline{2-3} 
\multicolumn{1}{l|}{}  & \multicolumn{1}{c|}{1} & \multicolumn{1}{c|}{2} \\ \cline{2-3} 
\end{tabular}
\caption{The payoff matrix for Example \ref{E:domstrat1} using only Player 2's payoffs}
\label{T:smallexampleP2}
\end{table}



\item Both players can make strategy decisions by considering only Player 1's payoff matrix. (Why?) Just to test this out, by looking only at the matrix in Table \ref{T:smallexampleP1} determine which strategy each player should choose.

%\hspace{.8in}Player 2

%\begin{tabular}{l|c|c|}\cline{2-3}
%Player 1&1&0\\ \cline{2-3}
%&-1&-2\\ \cline{2-3}

%\end{tabular}
%\medskip
\begin{table}[h]
\centering

\begin{tabular}{ccc}
                      & \multicolumn{2}{c}{Player 2}                                                  \\ \cline{2-3} 
%\multicolumn{1}{l|}{} & \multicolumn{1}{l|}{} & \multicolumn{1}{c|}{H} & \multicolumn{1}{c|}{T} \\ \cline{2-4} 
\multicolumn{1}{l|}{Player 1}  & \multicolumn{1}{c|}{1} & \multicolumn{1}{c|}{0} \\ \cline{2-3} 
\multicolumn{1}{l|}{}  & \multicolumn{1}{c|}{-1} & \multicolumn{1}{c|}{-2} \\ \cline{2-3} 
\end{tabular}
%\caption{Example \ref{E:domstrat1} using only Player 1's payoffs}
%\label{T:smallexampleP1}
\end{table}

 

\end{enumerate}

In this last example, it should be clear that Player 1 is looking for rows which give her the largest payoff-- this is nothing new. However, Player 2 is now looking for columns which give Player 1 the SMALLEST payoff. (Why?) 

Now that we have simplified our notation for zero-sum games, let's try to find a way to determine the best strategy for each player.

\begin{example}\label{E:domstrat2} Consider the zero-sum game given in Table \ref{T:biggerexampleP1}.

%\hspace{1in}Player 2

%\begin{tabular}{l|c|c|c|}\cline{2-4}
%Player 1&1&0&2\\ \cline{2-4}
%&-1&-2&2\\ \cline{2-4}

%\end{tabular}
%\medskip
\begin{table}[h]
\centering

\begin{tabular}{cccc}
                      & \multicolumn{3}{c}{Player 2}                                                  \\ \cline{2-4} 
%\multicolumn{1}{l|}{} & \multicolumn{1}{l|}{} & \multicolumn{1}{c|}{X} & \multicolumn{1}{c|}{Y} \\ \cline{2-4} 
\multicolumn{1}{l|}{Player 1} & \multicolumn{1}{c|}{1} & \multicolumn{1}{c|}{0} & \multicolumn{1}{c|}{2} \\ \cline{2-4} 
\multicolumn{1}{l|}{} & \multicolumn{1}{c|}{-1} & \multicolumn{1}{c|}{-2} & \multicolumn{1}{c|}{2} \\ \cline{2-4} 
\end{tabular}
\caption{Payoff matrix for Example \ref{E:domstrat2}}
\label{T:biggerexampleP1}
\end{table}


Determine which row Player 1 should choose. Is there any situation in which Player 1 would choose the other row? 
\end{example}



\begin{example}\label{E:domstrat3} Consider the zero-sum game given in Table \ref{T:biggerexample2}.

%\hspace{1in}Player 2

%\begin{tabular}{l|c|c|c|}\cline{2-4}
%Player 1&1&0&2\\ \cline{2-4}
%&-1&-2&3\\ \cline{2-4}

%\end{tabular}
%\medskip
\begin{table}[h]
\centering

\begin{tabular}{cccc}
                      & \multicolumn{3}{c}{Player 2}                                                  \\ \cline{2-4} 
%\multicolumn{1}{l|}{} & \multicolumn{1}{l|}{} & \multicolumn{1}{c|}{X} & \multicolumn{1}{c|}{Y} \\ \cline{2-4} 
\multicolumn{1}{l|}{Player 1} & \multicolumn{1}{c|}{1} & \multicolumn{1}{c|}{0} & \multicolumn{1}{c|}{2} \\ \cline{2-4} 
\multicolumn{1}{l|}{} & \multicolumn{1}{c|}{-1} & \multicolumn{1}{c|}{-2} & \multicolumn{1}{c|}{3} \\ \cline{2-4} 
\end{tabular}
\caption{Payoff matrix for Example \ref{E:domstrat3}}
\label{T:biggerexample2}
\end{table}



Determine which row Player 1 should choose. Is there any situation in which Player 1 would choose the other row? 
\end{example}


In Example \ref{E:domstrat2}, no matter what Player 2 does, Player 1 would always choose Row 1, since every payoff in Row 1 is greater than or equal to the corresponding payoff in Row 2 ($1\ge -1$, $0\ge -2$, $2\ge 2$). In Example \ref{E:domstrat3}, this is not the case: if Player 2 were to choose Column 3, then Player 1 would prefer Row 2. In Example \ref{E:domstrat2} we would say that Row 1 {\it dominates}\index{dominates} Row 2.


\begin{definition} A strategy $X$ \emph{dominates}\index{dominates} a strategy $Y$ if every entry for $X$ is greater than or equal to the corresponding entry for $Y$. In this case, we say $Y$ is \emph{dominated by}\index{dominated by} $X$.
\end{definition}

In mathematical notation: The $i^{\rm th}$ row dominates the  $j^{\rm th}$ row if $a_{ik}\ge a_{jk}$ for all $k$, and $a_{ik}> a_{jk}$ for at least one $k$. If $X$ dominates $Y$, we can write $X\succ Y$.

This definition can also be used for Player 2: we consider columns instead of rows. If we are looking at Player 1's payoffs, then Player 2 prefers smaller payoffs. Thus one column $X$ dominates another column $Y$ if all the entries in $X$ are smaller than or equal to the corresponding entries in $Y$.  

Here is the great thing: we can always eliminate dominated strategies! (Why?)
Thus, in Example \ref{E:domstrat2}, we can eliminate Row 2, as in Table \ref{T:strikerow2Ex2}.

%\hspace{1in}Player 2

%\begin{tabular}{l|c|c|c|}\cline{2-4}
%Player 1&1&0&2\\ \cline{2-4}
%&-1&-2&2\\ \cline{2-4}

%\end{tabular}
\begin{table}[h]
\centering

\begin{tabular}{cccc}
                      & \multicolumn{3}{c}{Player 2}                                                  \\ \cline{2-4} 
%\multicolumn{1}{l|}{} & \multicolumn{1}{l|}{} & \multicolumn{1}{c|}{X} & \multicolumn{1}{c|}{Y} \\ \cline{2-4} 
\multicolumn{1}{l|}{Player 1} & \multicolumn{1}{c|}{1} & \multicolumn{1}{c|}{0} & \multicolumn{1}{c|}{2} \\ \cline{2-4} 

\multicolumn{1}{l|}{} &  \multicolumn{1}{c|}{\MyTikzmark{leftA}{$-1$}} & \multicolumn{1}{c|}{-2} & \multicolumn{1}{c|}{\MyTikzmark{rightA}{2}}\\ \cline{2-4} 

\end{tabular}
\caption{Row 2 is dominated by Row 1}
\label{T:strikerow2Ex2}
\end{table}

\DrawHLine[blue, thick, opacity=0.5]{leftA}{rightA}

%\begin{picture}(0,0)
%\put(-70,-5){\line(10,0){72}}
%\end{picture}

 Now it is easy to see what Player 2 should do.
 
 In Example \ref{E:domstrat3}, we cannot eliminate Row 2 since it is not dominated by Row 1. However, it should be clear that Column 2 dominates Column 3 (remember, Player 2 prefers SMALLER values). Thus we can eliminate Column 3 as in Table \ref{T:strikecol3Ex3}.
 
% \hspace{1in}Player 2

%\begin{tabular}{l|c|c|c|}\cline{2-4}
%Player 1&1&0&2\\ \cline{2-4}
%&-1&-2&3\\ \cline{2-4}

%\end{tabular}
%\begin{picture}(0,0)
%\put(-12,-12){\line(0,1){32}}
%\end{picture}

\begin{table}[h]
\centering

\begin{tabular}{cccc}
                      & \multicolumn{3}{c}{Player 2}                                                  \\ \cline{2-4} 
%\multicolumn{1}{l|}{} & \multicolumn{1}{l|}{} & \multicolumn{1}{c|}{X} & \multicolumn{1}{c|}{Y} \\ \cline{2-4} 
\multicolumn{1}{l|}{Player 1} & \multicolumn{1}{c|}{1} & \multicolumn{1}{c|}{0} & \multicolumn{1}{c|}{\MyTikzmark{topB}{2}} \\ \cline{2-4} 

\multicolumn{1}{l|}{} &  \multicolumn{1}{c|}{-1} & \multicolumn{1}{c|}{-2} & \multicolumn{1}{c|}{\MyTikzmark{bottomB}{3}}\\ \cline{2-4} 

\end{tabular}
\caption{Column 3 is dominated by Column 2}
\label{T:strikecol3Ex3}
\end{table}

\DrawVLine[blue, thick, opacity=0.5]{topB}{bottomB}

%\medskip
 AFTER eliminating Column 3, Row 1 dominates Row 2. Now, in Table \ref{T:strikerow2Ex3} we can eliminate Row 2.
 
% \hspace{1in}Player 2

%\begin{tabular}{l|c|c|c|}\cline{2-4}
%Player 1&1&0&2\\ \cline{2-4}
%$&-1&-2&3\\ \cline{2-4}

%\end{tabular}
%\begin{picture}(0,0)
%\put(-12,-12){\line(0,1){32}}
%\put(-70,-5){\line(1,0){50}}
%\end{picture}
%\medskip

\begin{table}[h]
\centering

\begin{tabular}{cccc}
                      & \multicolumn{3}{c}{Player 2}                                                  \\ \cline{2-4} 
%\multicolumn{1}{l|}{} & \multicolumn{1}{l|}{} & \multicolumn{1}{c|}{X} & \multicolumn{1}{c|}{Y} \\ \cline{2-4} 
\multicolumn{1}{l|}{Player 1} & \multicolumn{1}{c|}{1} & \multicolumn{1}{c|}{0} & \multicolumn{1}{c|}{\MyTikzmark{topC}{2}} \\ \cline{2-4} 

\multicolumn{1}{l|}{} &  \multicolumn{1}{c|}{\MyTikzmark{leftD}{$-1$}} & \multicolumn{1}{c|}{-2} & \multicolumn{1}{c|}{\MyTikzmark{bottomC}{3}}\\ \cline{2-4} 

\end{tabular}
\caption{After eliminating Column 3, Row 2 is dominated by Row 1}
\label{T:strikerow2Ex3}
\end{table}

\DrawVLine[blue, thick, opacity=0.5]{topC}{bottomC}
\DrawHLine[red, thick, opacity=0.5]{leftD}{bottomC}

Again, now it is easy to determine what each player should do.

\begin{xca}
Check that the strategy pairs we determined in Examples \ref{E:domstrat2} and \ref{E:domstrat3} are, in fact, equilibrium pairs.
\end{xca}

\begin{xca}
Use the idea of eliminating dominated strategies to determine what each player should do in the Arnold/ Bainbridge examples from \ref{Ex:election} in Exercises \ref{E:election1}, \ref{E:election2}, and \ref{E:election3}. Do you get the same strategy pairs as you determined in those exercises?
\end{xca}


 