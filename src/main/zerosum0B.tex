
\subsection{Understanding the Players}

%\markright{Understanding the Assumptions}

\vspace{.2in}
%Do all of your work on your own paper. Give complete answers (complete sentences!).

%\vspace{.1in}

Recall that we said there are two major assumptions we must make about our players:
\begin{itemize}
\item Our players are \emph{self-interested}\index{self-interested}. This means they will always prefer the largest  possible payoff. They will choose a strategy which maximizes their payoff.
\item Our players are \emph{perfectly logical}\index{perfectly logical}. This means they will use all the information available and make the wisest choice for themselves.
\end{itemize}
It is important to note that each player also knows that his or her opponent is also self-interested and perfectly logical!

\begin{xca}
\begin{enumerate}[(a)]
%\item Which payoff does the player prefer: 10, 5, or 1?
\item Which payoff does the player prefer: 0, 2, or -2?
\item Which payoff does the player prefer: -2, -5, or -10?
\item Which payoff does the player prefer: -1, -3, or 0?
\end{enumerate}
\end{xca}
\vspace{.1in}

The real work begins when there are two players since Player 1 can only choose the row and Player 2 can only choose the column. Thus the outcome depends on BOTH players. 

%\begin{enumerate}
%\setcounter{enumi}{1}

\begin{xca}\label{E:Sec2.2small} Suppose two players are playing a game in which they can choose A or B with the payoffs given in the game matrix in Table \ref{T:matrixEx1Sec2.2}.

%\hspace{1.2in}Player 2
%\begin{tabular}{l|r|c|c|}\cline{2-4}
%&&\textbf{X}&\textbf{Y}\\ \cline{2-4}
%Player 1&\textbf{A} &(100, -100)&(-10, 10)\\ \cline{2-4}
%&\textbf{B} &(0, 0)&(-1, 1)\\ \cline{2-4}
%\end{tabular}

\begin{table}[h]
\centering

\begin{tabular}{cccc}
                      & \multicolumn{3}{c}{Player 2}                                                  \\ \cline{2-4} 
\multicolumn{1}{l|}{} & \multicolumn{1}{l|}{} & \multicolumn{1}{c|}{A} & \multicolumn{1}{c|}{B} \\ \cline{2-4} 
\multicolumn{1}{l|}{Player 1} & \multicolumn{1}{c|}{A} & \multicolumn{1}{c|}{$(100, -100)$} & \multicolumn{1}{c|}{$(-10, 10)$} \\ \cline{2-4} 
\multicolumn{1}{l|}{} & \multicolumn{1}{c|}{B} & \multicolumn{1}{c|}{$(0, 0)$} & \multicolumn{1}{c|}{$(-1, 11)$} \\ \cline{2-4} 
\end{tabular}
\caption{Payoff matrix for Exercise \ref{E:Sec2.2small}}
\label{T:matrixEx1Sec2.2}
\end{table}

\begin{enumerate}[(a)]
\item Just by quickly looking at the matrix, which player appears to be able to win more than the other player? Does one player seem to have an advantage? Explain.
\item Determine what each player should do. Explain your answer.
\item Compare your answer in (b) to your answer in (a). Did the player you suggested in (a) actually win more than the other player?
\item According to your answer in (b), does Player 1 end up with the largest possible payoff (for Player 1) in the matrix?
\item According to your answer in (b), does Player 2 end up with the largest possible payoff (for Player 2) in the matrix?
\item Do you still think a player has an advantage in this game? Is it the same answer as in (a)?
\end{enumerate}
\end{xca}
\vspace{.5in}

\begin{xca}\label{E:Sec2.2large}  Suppose there are two players with the game matrix given in Table \ref{T:matrixEx2Sec2.2}.

%\vspace{.1in}

%\hspace{1.3in}Player 2

%\begin{tabular}{l|r|c|c|c|}\cline{2-5}
%&&\textbf{X}&\textbf{Y}&\textbf{Z}\\ \cline{2-5}
%Player 1&\textbf{A} &(1000, -1000)&(-5, 5)&(-15, 15)\\ \cline{2-5}
%&\textbf{B} &(200, -200)&(0, 0)&(-5, 5)\\ \cline{2-5}
%&\textbf{C} &(500, -500)&(20, -20)&(-25, 25)\\ \cline{2-5}
%\end{tabular}

\begin{table}[h]
\centering

\begin{tabular}{ccccc}
                      & \multicolumn{4}{c}{Player 2}                                                  \\ \cline{2-5} 
\multicolumn{1}{l|}{} & \multicolumn{1}{l|}{} & \multicolumn{1}{c|}{A} & \multicolumn{1}{c|}{B} & \multicolumn{1}{c|}{C}\\ \cline{2-5} 
\multicolumn{1}{l|}{Player 1} & \multicolumn{1}{c|}{A} & \multicolumn{1}{c|}{$(1000, -1000)$} & \multicolumn{1}{c|}{$(-5, 5)$} & \multicolumn{1}{c|}{$(-15,15)$}\\ \cline{2-5} 
\multicolumn{1}{l|}{} & \multicolumn{1}{c|}{B} & \multicolumn{1}{c|}{$(200, -200)$} & \multicolumn{1}{c|}{$(0, 0)$} & \multicolumn{1}{c|}{$(-5,5)$}\\ \cline{2-5} 
\multicolumn{1}{l|}{} & \multicolumn{1}{c|}{C} & \multicolumn{1}{c|}{$(500, -500)$} & \multicolumn{1}{c|}{$(20, -20)$} & \multicolumn{1}{c|}{$(-25,25)$} \\ \cline{2-5} 
\end{tabular}
\caption{Payoff matrix for Exercise \ref{E:Sec2.2large}}
\label{T:matrixEx2Sec2.2}
\end{table}



\begin{enumerate}[(a)]
\item Just by quickly looking at the matrix, which player appears to be able to win more than the other player? Does one player seem to have an advantage? Explain.
\item Determine what each player should do. Explain your answer.
\item Compare your answer in (b) to your answer in (a). Did the player you suggested in (a) actually win more than the other player?
\item According to your answer in (b), does Player 1 end up with the largest possible payoff (for Player 1) in the matrix?
\item According to your answer in (b), does Player 2 end up with the largest possible payoff (for Player 2) in the matrix?
\item Do you still think a player has an advantage in this game? Is it the same answer as in (a)?
\end{enumerate}
\end{xca}

\vspace{.5in}

This chapter introduced you to who the players are and how to organize strategies and payoffs into a matrix. In the next chapter we will study some methods for how a player can determine his or her best strategy.

 